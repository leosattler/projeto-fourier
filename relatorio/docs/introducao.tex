%%%%%%%%%%%%%%%%%%%%%%%%%%%%%%%%%%%%%%%%%%%%%%%%%%%%%%%%%%%%%%%%%%%%%%%%%%%%%%%

\chapter{INTRODUÇÃO}

O fluxo solar na faixa de 10.7 cm (doravante chamado F10.7) é uma medida da intensidade da emissão do sol na faixa do rádio, mais precisamente em 10.7 cm (ou 2800 MHz). Este índice é um indicador da atividade magnética do Sol, fornecendo informações da atividade solar no ultravioleta e raio-X. Por isso, esse índice é muito relevante em ramos como astrofísica, meteoroglogia e geofísica. Com aplicações em modelagem climática, seu monitoramento é importante para a manutenção dos sistemas de comunicação via satélite \cite{huang2009forecast}. 

Em conjunto com os dados de manchas solares, F10.7 é um dos indicadores mais usados para previsão da atividade solar. Por esse motivo, muitos estudos objetivando predição do clima espacial o utilizam como parâmetro de input. Por exemplo, \citeonline{mordvinov1986prediction} utilizou autorregressão multiplicativa para predição mensal dos valores de F10.7. \citeonline{dmitriev1999solar} aplicaram redes neurais para a predição. Por sua vez, \citeonline{zhong2005application} aplicou análise espectral para prever os valores de F10.7. Já \citeonline{bruevich2014study} aplicou análise Wavelet sobre as médias mensais desse dado para análise da série temporal.

A análise espectral é um método para representar um sinal como a combinação linear de funções periódicas. Ela faz parte de uma família de técnicas chamadas de Análise de Fourier. No presente trabalho, os dados do índice F10.7 são analisados no contexto da Análise de Fourier. Este manuscrito está assim organizado: na Seção 2 os dados e os tratamentos nele realizados são descritos; na Seção 3 é feita uma recapitulação da Análise de Fourier; na Seção 4 os resultados são apresentados com uma breve discussão; na Seção 5 são oferecidas as considerações finais do autor.

%%%%%%%%%%%%%%%%%%%%%%%%%%%%%%%%%%%%%%%%%%%%%%%%%%%%%%%%%%%%%%%%%%%%%%%%%%%%%%%%
% RESUMO %% obrigatório

\begin{resumo}

%% neste arquivo resumo.tex
%% o texto do resumo e as palavras-chave têm que ser em Português para os documentos escritos em Português
%% o texto do resumo e as palavras-chave têm que ser em Inglês para os documentos escritos em Inglês
%% os nomes dos comandos \begin{resumo}, \end{resumo}, \palavraschave e \palavrachave não devem ser alterados

\hypertarget{estilo:resumo}{} %% uso para este Guia

Neste trabalho, dados do fluxo solar na faixa de 10.7 cm foram analisados no contexto da Análise de Fourier. Três conjuntos de dados, médias diárias, médias de 27 dias e médias anuais foram explorados através da biblioteca \texttt{numpy} (da linguagem \texttt{Python}) com a rotina \texttt{numpy.fft}. O espectro de potência dos sinais foi extraído e seu resultado discutido à luz do conhecido ciclo de atividade solar, cujo período é de onze anos. O objetivo deste projeto é a familiarização de técnicas para análise de séries temporais através do formalismo de Fourier, contribuindo para a assimilação de conceitos relevantes à disciplina Análise Wavelet I.

\palavraschave{%
	\palavrachave{Fluxo solar}%
	\palavrachave{Transformada de Fourier}%
	\palavrachave{Espectro de potência}%
	\palavrachave{Análise de sinal}%
	\palavrachave{Séries temporais}%
}
 
\end{resumo}
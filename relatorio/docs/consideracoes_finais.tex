%%%%%%%%%%%%%%%%%%%%%%%%%%%%%%%%%%%%%%%%%%%%%%%%%%%%%%%%%%%%%%%%%%%%%%%%%%%%%%%

\chapter{CONSIDERAÇÕES FINAIS}

As atividades realizadas no presente trabalho tiveram como objetivo solidificar os conceitos pertinentes à análise de sinais temporais. As ferramentas da Análise de Fourier foram aplicadas em dados de fluxo solar na faixa de 10.7 cm, que indicam a atividade solar cujo ciclo é conhecido e igual a onze anos. Dados de diferentes tamanhos referentes ao mesmo período temporal foram adquiridos e, em seguida: (1) explorados para familiarização e identificação de inconsistências; (2) tratados de maneira pertinente à análise espectral e (3) analisados com a rotina \texttt{numpy.fft} do \texttt{Python}. 

A rotina \texttt{numpy.fft} forneceu a Transformada Discreta de Fourier através do algoritmo FFT (Fast Fourier Transform). O resultado foi utilizado para gerar o espectro de potência. As Figuras 4.1 e 4.2 sugerem que a análise foi capaz de apontar o ciclo de período esperado (onze anos). Tais resultados indicam que as ferramentas utilizadas foram assimiladas de maneira satisfatória.
